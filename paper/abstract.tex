\begin{abstract}
Distributed OLTP systems 
execute the high-overhead,
  \emph{two-phase commit} (2PC) protocol at the end of every distributed transaction.
  \emph{Epoch-based commit} proposes that 2PC be executed only once for all transactions processed within a time interval called an \emph{epoch}.
  Increasing epoch duration allows more transactions to be processed before the common 2PC. It thus reduces 2PC overhead per transaction, increases throughput but also increases average   transaction latency.  
  Therefore, required is the ability to choose the right epoch size that offers the desired trade-off between throughput and latency.
To this end, we develop two analytical models to estimate throughput and average latency in terms of epoch size taking into account load and failure conditions. 
Simulations affirm their accuracy and effectiveness. We then present
\emph{epoch-based multi-commit}
which, unlike epoch-based commit, seeks to avoid all transactions being aborted when failures occur, and also performs identically when failures do not occur. Our performance study identifies workload factors that make it more effective in preventing transaction aborts and concludes that the analytical models can be equally useful in predicting its performance as well. 
%but only those that accessed data in the failed node, directly or indirectly.
%We next focus on a performance-unfriendly aspect of epoch-based commit: \emph{all} transactions of an epoch abort in case of a node failure even though some never accessed any data in the failed node, directly or transitively. 
%   To avoid these unnecessary aborts,
%   we propose  \emph{epoch-based multi-commit} variation.
%   An extensive, comparative evaluation 
%   between the two versions 
%   is carried out using simulations.
%   Certain workload 
%   characteristics
%   %, e.g., a small proportion of distributed transactions,
%   are found to
%   strongly favour the use of multi-commit which defaults automatically to epoch-based commit in all  unfavourable situations. So, no workload analysis is necessary prior to its deployment. 
  
  %In epoch-based commit, when a database node fails, all transactions within 
  %an epoch are aborted, resulting in a significant amount of wasted work, which must be
  %re-executed.
  %Given certain workload characteristics, namely, the proportion of distributed transactions,
  %the number of remote servers accessed, and node affinity, we observe this can result in unnecessarily discarding completed work. 
  %Thus, we present \emph{epoch-based multi-commit}, an atomic commitment
  %protocol that leverages these workload characteristics in order to increase resiliency
  %to node failures, decreasing wasted work. 
 
\end{abstract}

\begin{IEEEkeywords}
  Distributed Databases, Transactions, Two-Phase Commit, Epochs, Analytical solutions, Simulations, Performance Evaluation, Throughput, Latency
\end{IEEEkeywords}
