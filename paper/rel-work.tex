\section{Related Work}
\label{sec:related-work}

% 1: approaches to improve distributed transaction processing;
% (i) minimize/eliminate, (ii) determinism, (iii) relax consistency, (iv) blended approaches, (v) epochs
Numerous approaches to improve distributed transaction processing performance have been proposed.
%Several systems attempt to \emph{minimize} or \emph{eliminate} distributed transactions.
Workload-driven partitioning can \emph{minimize} distributed transactions~\cite{curino} and dynamic data repartitioning can \emph{eliminate} them completely~\cite{das,lin}.
% Schism uses a workload-driven partitioning scheme to \emph{minimize}  
% distributed transactions~\cite{curino}, while
% G-Store~\cite{das} and LEAP~\cite{lin} \emph{eliminate} them through dynamic data repartitioning.
% Others avoid 2PC by \emph{mandating determinism}~\cite{thomson,ren,thomson2}.
\emph{Mandating determinism} also avoids 2PC~\cite{thomson,ren,thomson2}.
%The idea behind deterministic databases is to order and replicate transactional inputs prior to execution using a sequencing layer.
% Pioneers of this approach are Calvin~\cite{thomson} and SLOG~\cite{ren}.
A caveat with such a mandate is that 
%deterministic 
% databases is 
%that 
transactions' read and write sets be known prior 
to execution~\cite{ren2},
%If it cannot be met, 
else, a reconnaissance phase is executed to
discover these sets which amounts to running a transaction twice. 
Aria~\cite{lu2} avoids this caveat by using a deterministic reordering mechanism, but its performance suffers under high contention workloads~\cite{lu}.
Prognosticator~\cite{issa} circumvents this limitation by using symbolic execution to build key-level 
transaction profiles.
%, which are used to execute transactions with a high degree of parallelism.

% Another category of systems 
Some systems
% \emph{weaken isolation} and 
\emph{relax consistency guarantees} 
% \emph{consistency guarantees} 
to achieve 
better performance, offering snapshot isolation~\cite{elnikety,du,sovran} or highly available 
transactions~\cite{bailis}.
% Another group of systems 
%opt for a \emph{blended approach} by 
Others
combine concurrency control, replication, and commitment into a unified protocol 
to reduce WAN round trip instances and thus decrease latency
% to amortize the costs of 2PC.
% Thereby reducing the number of WAN RTs and hence latency
% round trips 
% ~\cite{kraska,zhang,fan,mu,nawab,maiyya}.
% Thereby 
%principal aim here is to minimize 
% the number of WAN round trips and hence latency are reduced
~\cite{kraska,zhang,fan,mu,nawab,maiyya}.
%decrease latency.
% (see MDCC~\cite{kraska}, TAPIR~\cite{zhang}, Ocean Vista~\cite{fan},
% Janus~\cite{mu}, Helios~\cite{nawab}, and G-PAC~\cite{maiyya}).
Similarly, \emph{Parallel Commits}~\cite{taft} 
% is a related technique that 
halves latency by concurrently performing
consensus round trips.
% Another related technique~\cite{taft}
%is CockroachDB~\cite{taft}
% uses \emph{Parallel Commits} to halve the latency 
%of distributed transactions 
% by concurrently performing
% consensus round trips. % required to commit a transaction. 

Several 
% recent 
OLTP databases utilize epochs to exchange improved throughput
for higher latency.
Silo 
uses epochs to
% in the context of 
% a single-node many-core database,
%does not detect anti-dependencies within epochs in order to 
reduce shared memory writes~\cite{tu}.
Obladi %tackles the problem of data access privacy by 
combines 
them with Oblivious RAM 
% and
% epochs 
to hide access patterns~\cite{crooks}.
%from cloud providers
STAR~\cite{lu3} runs distributed 
% transactions 
and single-node transactions in different
epochs.
COCO~\cite{lu} leverages epochs to mitigate against the costs of 2PC and
synchronous replication.
%To our knowledge,
Our epoch-based multi-commit is the first to combine
epochs and data access patterns to minimise aborts in the presence of node
failures.

% 2: analytical modelling
Constructing analytical models of data management systems has a rich history.
Objectives vary: performance prediction in messaging systems~\cite{wu}, 
%analysing consistency properties % of blockchain protocols
%~\cite{meng},
%Analytical models have been used extensively to 
studying the behavior of concurrency control protocols %for transaction-processing systems
~\cite{yu,agrawal,disanzo} 
%and for Software Transactional Memories
, and
%They have also been extensively utilised
%in the study of 
atomic commitment protocols in distributed 
databases~\cite{menasce,liu}.
Our work is the first to  %builds on previous work, 
present analytical models for epoch-based commit.


