\section{Conclusion}
\label{sec:conclusion}

We have developed two analytical models for the epoch-based commit protocol which allow database operators to 
maximize throughput, minimize average response time, or seek a trade-off between them. The accuracy of these 
models has been validated through a simulation study that considered a cluster of 64 nodes operating for 100 
days. We also developed epoch-based multi-commit, which aims to minimise transaction aborts in the event of node 
failures, but performs identically to the original version under other circumstances. Our simulation study 
affirms that multi-commit performs better when distributed transactions originating at a given node tend to 
access specific other nodes in their remote interactions. 
%The workload characteristics for which multi-commit decrease lost transactions were explored and identified.
%When node affinity is high, the number of distributed transaction does not dominant the
%workload, and each access a small number of nodes, the lost transaction rate can be decreased by up to 83\%.
When failures are rare, the analytical expressions derived for the original protocol can also be used in 
determining the right epoch intervals for the multi-commit version as well. Thus, we offer a practical 
alternative to epoch-based commit and analytical solutions to efficiently tune the parameter of epoch-based 
commit protocols in practical settings.  

%An interesting avenue for future work is investigating how the epoch coordinator can be decentralized.
%This could help mitigate against imbalanced workloads and stragglers by allowing
%nodes to move through epochs locally and only forming commit groups when needed.
%This could be appealing for geo-distributed deployments.
